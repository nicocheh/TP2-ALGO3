\documentclass[A4paper,oneside,fleqn,10pt]{article}

% This first part of the file is called the PREAMBLE. It includes
% customizations and command definitions. The preamble is everything
% between \documentclass and \begin{document}.

%Cambiamos un poquito los márgenes%
\addtolength{\oddsidemargin}{-1in}
\addtolength{\evensidemargin}{-1in}
\addtolength{\textwidth}{2in}
\addtolength{\topmargin}{-1in}
\addtolength{\textheight}{2in}


\usepackage{graphicx}              % to include figures
\usepackage{amsmath}               % great math stuff
\usepackage{amsmath,scalerel}
\usepackage{amsfonts}              % for blackboard bold, etc
\usepackage{amsthm}                % better theorem environments
\usepackage{amssymb}
\usepackage{mathrsfs}
\usepackage[spanish]{babel}
\usepackage[utf8]{inputenc}
\usepackage{hyperref}
\usepackage{multicol}
\usepackage{tikz-cd}
\usepackage{amsmath}
\usepackage[linesnumbered,ruled]{algorithm2e}
\usepackage{algpseudocode}
\usetikzlibrary{calc}
\usetikzlibrary{matrix}
\usepackage{graphicx,wrapfig,lipsum}

\graphicspath{ {Grphs/} }


\setcounter{tocdepth}{3}% to get subsubsections in toc

\let\oldtocsection=\tocsection

\let\oldtocsubsection=\tocsubsection

\let\oldtocsubsubsection=\tocsubsubsection

% various theorems, numbered by section

\newtheorem{teo}{Teorema}[section]
\newtheorem{lem}[teo]{Lema}
\newtheorem{prop}[teo]{Proposición}
\newtheorem{cor}[teo]{Corolario}
\newtheorem{crit}[teo]{Criterio}
\newtheorem{propi}[teo]{Propiedad}

\theoremstyle{definition}
\newtheorem{ejcio}[teo]{Ejercicio}
\newtheorem{conj}[teo]{Conjetura}
\newtheorem{obs}[teo]{Observación}
\newtheorem{defn}[teo]{Definición}
\newtheorem{ax}[teo]{Axioma}
\newtheorem{ex}[teo]{Ejemplo}

\newcommand{\bd}[1]{\mathbf{#1}}  % for bolding symbols
\newcommand{\cl}[1]{\overline{#1}} 
\newcommand{\CC}{\mathbb{C}}
\newcommand{\RR}{\mathbb{R}}      % for Real numbers
\newcommand{\ZZ}{\mathbb{Z}}      % for Integers
\newcommand{\NN}{\mathbb{N}}
\newcommand{\QQ}{\mathbb{Q}}
\newcommand{\FF}{\mathbb{F}}
\newcommand{\col}[1]{\left[\begin{matrix} #1 \end{matrix} \right]}
\newcommand{\comb}[2]{\binom{#1^2 + #2^2}{#1+#2}}
\newcommand{\eps}{\varepsilon}
\renewcommand{\hom}{\mathrm{Hom}}
\let\oldemptyset\emptyset
\let\emptyset\varnothing
\DeclareMathOperator{\id}{id}
\DeclareMathOperator{\mcm}{mcm}
\DeclareMathOperator{\mcd}{mcd}
\DeclareMathOperator{\ord}{ord}
\DeclareMathOperator{\im}{im}
\DeclareMathOperator{\End}{End}
\DeclareMathOperator{\Aut}{Aut}
\DeclareMathOperator{\sg}{sg}
\DeclareMathOperator{\cok}{cok}
\DeclareMathOperator{\ext}{Ext}
\DeclareMathOperator{\Obj}{Obj}
\DeclareMathOperator{\rank}{rk}
\DeclareMathOperator{\gr}{gr}
\DeclareMathOperator{\car}{char}
\DeclareMathOperator{\Nil}{Nil}
\DeclareMathOperator{\spec}{Spec}
\DeclareMathOperator{\ev}{ev}
\DeclareMathOperator{\ann}{Ann}
\DeclareMathOperator{\tr}{Tr}
\DeclareMathOperator*{\bigcdot}{\scalerel*{\cdot}{\bigodot}}
\def\acts{\curvearrowright}
\def\stca{\curvearrowleft}

\setcounter{tocdepth}{10}
\setcounter{secnumdepth}{10}

\usepackage[utf8]{inputenc}
\usepackage{fancyhdr}

 \chead{Algo III, TP2, Chehebar, Duran, Somacal}
 
\title{Algoritmos y Estructuras de Datos III, TP2}
\author{Nicolás Chehebar, Matías Duran, Lucas Somacal}
\date{}


\begin{document}

\pagenumbering{roman}
\pagenumbering{arabic}
\maketitle
\tableofcontents
\clearpage

\section{Problema 1}
\subsection{El Problema}

\subsubsection{Descripcion}

\subsubsection{Ejemplos}



\subsection{El Algoritmo sin podas}

\subsubsection{Resumen}
\subsection{Complejidad}

\subsection{Experimentación}
\subsubsection{Contexto}

La experimentacion se realizó toda en la misma computadora, cuyo procesador era Intel(R) Atom(TM) CPU N2600 @ 1.60GHz, de 36 bits physical, 48 bits virtual, con una memoria RAM de 2048 MB.  Para experimentar, se calculó el tiempo que tardaba el algoritmo sin considerar el tiempo de lectura y escritura ni el tiempo que llevaba armar la matriz (ya que se leía un dato, se escribía la matriz y luego se leia el siguiente). El tiempo se medía no como tiempo global sino como tiempo de proceso, calculando la cantidad de ticks del reloj (con el tipo clock\_t de C++) y luego se dividìa el delta de ticks sobre CLOCKS\_PER\_SEC. En todos los experimentos el tiempo se mide en segundos. 

\subsection{Conclusiones}


\section{Problema 2.1}
\section{Problema 2.2}
\section{Problema 3}

\end{document}